 \documentclass[book.tex]{subfiles}
\begin{document}

In September 1990, John Carmack, developed his first version of \textit{Adaptive Tile Refreshment}. He discussed the idea with coworker Tom Hall, who encouraged him to demonstrate it by recreating the first level of the recent Super Mario Bros. 3 on a computer. The pair did so in a single overnight session, with Hall recreating the graphics of the game. They replaced the player character of Mario with Dangerous Dave, a character from an eponymous previous Gamer's Edge game, while Carmack optimized the code. The next morning on September 20, the resulting game, \textit{Dangerous Dave in Copyright Infringement}, was shown to their other coworker John Romero. \\

\par
\begin{fancyquotes}
As soon as the demo started running, I pressed the right arrow key to see if magic had indeed been made. As soon as little Dave walked a short way to the right...\\

THE SCREEN SCROLLED.\\

SMOOTHLY.\\

Time stopped.\\

I was speechless...\\
\par
\textbf{John Romero - co-founder of id Software.}
\end{fancyquotes}\\
\par



Romero recognized Carmack's idea as a major accomplishment: Nintendo was one of the most successful companies in Japan, largely due to the success of their Mario franchise, and the ability to replicate the gameplay of the series on a computer could have large implications.\\

\begin{figure}[H]
\centering
 \fullimage{dangerous_dave.png}
 \caption{Dangerous Dave in Copyright Infringement demo}
 \label{fig:ddici}
\end{figure}

\par
The manager of the team (who called themselves \textit{Ideas from the deep}) and fellow programmer, Jay Wilbur, recommended that they take the demo to Nintendo itself, to position themselves as capable of building a PC version of Super Mario Bros. for the company. The team (composed of Carmack, Romero, Hall, and Wilbur, along with Lane Roathe, the editor for Gamer's Edge) decided to build a full demo game for their idea to send to Nintendo. As they lacked the computers to build the project at home, and could not work on it at Softdisk, they "borrowed" their work computers over the weekend, taking them in their cars to a house shared by Carmack, Wilbur, and Roathe, and made a copy of the first level of the game over the next 72 hours. The team send the demo to Nintendo Of America to see if they could do the PC port of the game. \\

\par
The demo made it to Nintendo of Japan and Shigeru Miyamoto specifically. They were very impressed with the demo, but their corporate plan was to never release their IP on a platform other than their own.\\


\end{document}