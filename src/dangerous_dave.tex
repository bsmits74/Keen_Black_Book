 \documentclass[book.tex]{subfiles}
\begin{document}

\subsection{Dangerous Dave in Copyright Infringement}
In September 1990, John Carmack, developed his first version of \textit{Adaptive Tile Refreshment}. He discussed the idea with coworker Tom Hall, who encouraged him to demonstrate it by recreating the first level of the recent Super Mario Bros. 3 on a computer. The pair did so in a single overnight session, with Hall recreating the graphics of the game. They replaced the player character of Mario with Dangerous Dave, a character from an eponymous previous Gamer's Edge game, while Carmack optimized the code. The next morning on September 20, the resulting game, Dangerous Dave in Copyright Infringement, was shown to their other coworker John Romero. Romero recognized Carmack's idea as a major accomplishment: Nintendo was one of the most successful companies in Japan, largely due to the success of their Mario franchise, and the ability to replicate the gameplay of the series on a computer could have large implications.\\

\begin{figure}[H]
\centering
 \fullimage{dangerous_dave.png}
 \caption{Dangerous Dave in Copyright Infringement demo}
 \label{fig:ddici}
\end{figure}

\par
The manager of the team and fellow programmer, Jay Wilbur, recommended that they take the demo to Nintendo itself, to position themselves as capable of building a PC version of Super Mario Bros. for the company. The team (composed of Carmack, Romero, Hall, and Wilbur, along with Lane Roathe, the editor for Gamer's Edge) decided to build a full demo game for their idea to send to Nintendo. As they lacked the computers to build the project at home, and could not work on it at Softdisk, they "borrowed" their work computers over the weekend, taking them in their cars to a house shared by Carmack, Wilbur, and Roathe, and made a copy of the first level of the game over the next 72 hours. The team send the demo to Nintendo Of America to see if they could do the PC port of the game. The demo made it to Nintendo of Japan and Shigeru Miyamoto specifically. They were very impressed with the demo, but their corporate plan was to never release their IP on a platform other than their own.\\

\subsection{Founding of id Software}
Around the same time as the group was rejected by Nintendo, Romero was approached by Scott Miller of Apogee Software. They agreed to make \textit{Commander Keen in Invasion of the Vorticons}, to be published by Apogee Software. The team could not afford to leave their jobs to work on the game full-time, so they continued to work at Softdisk, spending their time on the Gamer's Edge games during the day and on Commander Keen at night and weekends using Softdisk computers. The game was completed in early December 1990.\\

\par
After the arrival of the first royalty check from Apogee, the team planned to quit Softdisk and start their own company. On February 1, 1991, the team founded \textit{id Software} having four owners: John Carmack, John Romero, Tom Hall and artist Adrian Carmack\footnote{See Masters of Doom, chapter 4}. When their boss and owner of Softdisk, Al Vekovius, confronted them on their plans, as well as their use of company resources to develop the game, the team made no secret of their intentions. Vekovius initially proposed a joint venture between the team and Softdisk, which fell apart when the other employees of the firm threatened to quit in response, and after a few weeks of negotiation the team agreed to produce a series of games for Gamer's Edge, one every two months. One of the games they developed to fulfill their obligation was Commander Keen in Keen Dreams.\\

\par
Between 1990 and 1991 the team published \textit{Commander Keen in Invasion of the Vorticons} and \textit{Commander Keen in Goodbye, Galaxy}, and the stand-alone games \textit{Commander Keen in Keen Dreams} and \textit{Commander Keen in Aliens Ate My Babysitter}. Another trilogy of episodes, titled \textit{The Universe Is Toast}, was planned for December 1992; id worked on it for a couple of weeks, but then shifted the work to another game. The name of that new game was \textbf{Wolfenstein 3D}...


\end{document}