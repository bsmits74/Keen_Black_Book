 \documentclass[book.tex]{subfiles}
\begin{document}
\label{sec:epilogue}

Upon release, Commander Keen in Invasion of the Vorticons was an immediate hit, generating over US\$60,000 per month. In the summer of 1991, id Software hosted a seminar for game developers with the intention of licensing the Commander Keen engine. This effort became a spiritual predecessor to id Software’s later, more formal practice of licensing its game engines.\\

\par
Although this licensing initiative was not as successful as the later idTech engines, the Commander Keen engine was used in several games, including:
\begin{itemize}
    \item Dangerous Dave
    \item Shadow Knights
    \item Bio Menace
    \item ScubaVenture
\end{itemize}

\par
A third trilogy of episodes, titled The Universe Is Toast, was planned for release in December 1992. id Software worked on it for a couple of weeks before shifting its focus to Wolfenstein 3D, and the project was never resumed.\\

\par
\begin{minipage}{.5\textwidth}
In 1999, John Carmack remarked that he was considering developing a Commander Keen game for the Game Boy Color. Sometime afterward, id Software approached Activision with the idea. Activision recommended David A. Palmer Productions to develop the game, with oversight from id Software. In 2001, ten years after the series’ first release, Commander Keen appeared on the Game Boy Color.\\
\end{minipage}
\begin{minipage}{.5\textwidth}
\vspace{-15pt} 
\begin{figure}[H]
  \centering
  \includegraphics[width=.75\linewidth]{screenshots_300dpi/KeenGBC-ingame.png}
\end{figure}
\end{minipage}

\vspace{10pt}

\par
In June 2019, during Bethesda’s E3 conference, a new Commander Keen game for iOS and Android devices was announced. It was planned for release in the summer of the same year; however, no release or further announcements followed, and by June 2020 all references to the game had been removed from Bethesda and ZeniMax websites.\\

\par
While the Commander Keen franchise declined after its initial success, id Software continued to thrive. In the years that followed, the company became one of the most successful and influential game development studios in the industry, largely due to the Wolfenstein 3D, Doom, and Quake franchises, as well as its pioneering work in game engine licensing. id Software was acquired by ZeniMax Media in 2009, which was subsequently acquired by Microsoft in 2020.


\end{document}