 \documentclass[book.tex]{subfiles}
\begin{document}
\label{sec:epilogue}

Upon release, Commander Keen in Invasion of the Vorticons was an immediate hit, generating over US\$60,000 per month. In the summer of 1991, id Software hosted a seminar for game developers with the intention of licensing the Commander Keen engine. This effort became a spiritual predecessor to id Software’s later, more formal practice of licensing its game engines.\\

\par
Although this licensing initiative was not as successful as the later idTech engines, the Commander Keen engine was used in several games, including:
\begin{itemize}
    \item Dangerous Dave
    \item Shadow Knights
    \item Bio Menace
    \item ScubaVenture
\end{itemize}

\par
A third trilogy of episodes, titled \textit{The Universe Is Toast}, was planned for release in December 1992. id Software worked on it for a couple of weeks before shifting its focus to \textit{Wolfenstein 3D}, and the project was never resumed.\\

 \begin{wrapfigure}[8]{l}{0.3\textwidth}
\centering
\vspace{-10pt}
\includegraphics[width=.3\textwidth]{screenshots_300dpi/KeenGBC-ingame.png}
\end{wrapfigure}
\par
In 1999, John Carmack was considering developing a Commander Keen game for the Game Boy Color. Sometime afterward, id Software approached Activision with the idea. Activision recommended David A. Palmer Productions to develop the game, with oversight from id Software. In 2001, ten years after the series’ first release, Commander Keen appeared on the Game Boy Color.\\


\par
In June 2019, during Bethesda’s E3 conference, a new Commander Keen game for iOS and Android devices was announced. It was planned for release in the summer of the same year; however, no release or further announcements followed, and by June 2020 all references to the game had been removed from Bethesda and ZeniMax websites.\\

\section{The Dopefish legacy}
While the Commander Keen franchise declined after its initial success, one character from the series created its own legacy: the Dopefish. It was created by Tom Hall and it first appeared in the "\textit{Well of Wishes}" level of Commander Keen 4.\\

\begin{fancyquotes}
The second-dumbest creature in the universe, this creature's thought patterns go "swim swim hungry, swim swim hungry." They'll eat anything alive and moving near them, though they prefer heroes.\\
\end{fancyquotes}\\

 \begin{wrapfigure}[8]{r}{0.3\textwidth}
\centering
\vspace{-10pt}
\includegraphics[width=.3\textwidth]{screenshots_300dpi/game/Dopefish.png}
\end{wrapfigure}
\par
Soon, the Dopefish began appearing in other games, often as a quiet in-joke: carved into walls in \textit{Doom}, hidden on computer screens in \textit{Quake}, floating on signs and textures in \textit{Quake II} and \textit{Quake III Arena}. Later, it surfaced in unexpected places, such as graffiti in \textit{Max Payne}, posters in \textit{Daikatana}, and as a hobblehead in \textit{Hitman 2}. Its most recent appearance, as of writing this book, is in \textit{Doom Eternal} (2020)\footnote{For all appearances, see https://www.dopefish.com/fishinfo.html.}.


\end{document}