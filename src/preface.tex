\documentclass[book.tex]{subfiles}
\begin{document}

I was 11 years old when I wrote my first lines of software code. It was on an MSX computer in BASIC, and it opened up a completely new, magical world for me. Three years later, I got my first PC---an Intel 80286. I "witnessed" the rise of the PC as a gaming platform and played almost all of the groundbreaking titles: \textit{Prince of Persia}, \textit{Wolfenstein 3D}, \textit{Doom}, \textit{Dune II}, \textit{Command \& Conquer}, and many more.\\

\par
But for me, it all started with an earlier game: \textit{Commander Keen}. It was the first time I saw smooth scrolling on a PC. It was also the first major title from id Software. Thanks to the financial success of Commander Keen, the team --- then still employed by Softdisk --- decided to start their own game development company. This is the same game studio that would later release the groundbreaking games Wolfenstein 3D and Doom.\\

\par
Fast forward to 2021, I discovered Fabien Sanglard’s website and began reading his Game Engine Black Books on Wolfenstein 3D and Doom. Inspired by those works, I wondered whether I could do something similar for Commander Keen: open up the source code, explore the files, and piece together a picture of the overall architecture and the clever tricks used. The title, style, and dimensions of this book are intentionally similar to Fabien’s Game Engine Black Books, as an homage to those masterpieces.\\

\par
The entire experience was great. It felt like returning to my early teenage years, debugging and experimenting with C and assembly code on a MS-DOS computer. I learned a great deal about the hardware of that era and the challenges developers faced in getting everything to work together.\\

\par
You might ask why I “wasted” my time on a game and hardware more than 35 years old. Well, even though we now have vastly more transistors on a chip and have gone from megahertz to gigahertz CPUs, from kilobytes to gigabytes of RAM, and from simple video cards to powerful GPUs, the underlying architecture remains surprisingly similar to what it was three decades ago.\\


\par
The result of my journey is in this book --- a mix of engineering, history, and nostalgia. It is written by a Dutch guy, so forgive my imperfect English (Microsoft Copilot was a great help in reviewing my style and grammar).  To compensate for my limitations in prose, I’ve included many detailed drawings to illustrate the hardware and software techniques discussed.\\

\par
So turn this page, sit back, and let’s travel together to the early ’90s.\\

\par
I hope you enjoy reading it.\\

-- Bas Smits\\
\par
Helmond, The Netherlands\\
\monthyeardate\today\\


\end{document}
